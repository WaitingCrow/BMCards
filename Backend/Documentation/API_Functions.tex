\documentclass[a4paper,12pt]{article}

\addtolength{\oddsidemargin}{-.875in}
\addtolength{\evensidemargin}{-.875in}
\addtolength{\textwidth}{1.75in}
\addtolength{\topmargin}{-.875in}
\addtolength{\textheight}{1.75in}

\title{\par\noindent\huge{API functions documentation}}
\date{21th, Oct 2022}
%\author{BMCards team}

\begin{document}

    \pagenumbering{arabic}
	\maketitle
	\newpage
%__________________________________________________________________________________________________________________________________

	\tableofcontents
	\newpage
%__________________________________________________________________________________________________________________________________

    \section[User API functions]{User API functions}
        
        \vspace{10pt}

        These functions concern the manipulation of the user database which is responsible for holding the users, user messages, friendships as well as other relations. The implemented abstractions behave in a stateless and are compliant with the REST design pattern. The following is the list with all the API functions:

        \subsection[User login]{User login}

            \vspace{10pt}

            This request, called \textbf{user\_login}, can be used to perform the login of the user by passing the JSON encoded credentials in the body of the request.

            \begin{itemize}

                \item \textit{url endpoint: \ldots/userservices/login.php}
                \item \textit{method: POST}
                \item \textit{possible HTTP status codes:}

                    \begin{itemize}

                        \item \textit{200 ``OK'': request was successful.}
                        \item \textit{400 ``Bad Request'': request had missing or malformed parameters.}
                        \item \textit{401 ``Unauthorized'': user doesn't exist.}
                        \item \textit{403 ``Forbidden'': wrong password for the specified username or user is already logged-in.}
                        \item \textit{404 ``Not found'': could not find the specified service.}
                        \item \textit{500 ``internal server error'': error in negotiation with the database.}

                    \end{itemize}

                \item \textit{request parameters: JSON object containing two keys.}
    
    \begin{verbatim}
                        
        {
            "username": String,
            "password": String
        }

    \end{verbatim}

                \item \textit{response parameters: JSON object containing different keys depending on the returned HTTP status code.}

    \begin{verbatim}
                        
        {
            "user_already_loggedin": (Boolean | null),
            "payload": {

                "user_id": (Int | null)
            }
        }
            
    \end{verbatim}

            \end{itemize}

        \clearpage
        \subsection[User register]{User register}

            \vspace{10pt}

            This request, called \textbf{user\_register}, can be used to perform the registration of the user by passing the JSON encoded credentials in the body of the request.

            \begin{itemize}

                \item \textit{url endpoint: \ldots/userservices/register.php}
                \item \textit{method: POST}
                \item \textit{possible HTTP status codes:}

                    \begin{itemize}

                        \item \textit{200 ``OK'': request was successful.}
                        \item \textit{400 ``Bad Request'': request had missing or malformed parameters.}
                        \item \textit{403 ``Forbidden'': user already exists.}
                        \item \textit{404 ``Not found'': could not find the specified service.}
                        \item \textit{500 ``internal server error'': error in negotiation with the database.}

                    \end{itemize}

                \item \textit{request parameters: JSON object containing two keys.}
    
    \begin{verbatim}
                        
        {
            "username": String,
            "password": String
        }

    \end{verbatim}

                \item \textit{response parameters: there is no response body.}

            \end{itemize}

        \clearpage
        \subsection[User logout]{User logout}

            \vspace{10pt}

            This request, called \textbf{user\_logout}, can be used to perform the logout of the user by passing the JSON encoded username in the body of the request.

            \begin{itemize}

                \item \textit{url endpoint: \ldots/userservices/logout.php}
                \item \textit{method: DELETE}
                \item \textit{possible HTTP status codes:}

                    \begin{itemize}

                        \item \textit{200 ``OK'': request was successful.}
                        \item \textit{400 ``Bad Request'': request had missing or malformed parameters.}
                        \item \textit{401 ``Unauthorized'': user doesn't exist.}
                        \item \textit{403 ``Forbidden'': user is already offline.}
                        \item \textit{404 ``Not found'': could not find the specified service.}
                        \item \textit{500 ``internal server error'': error in negotiation with the database.}

                    \end{itemize}

                \item \textit{request parameters: JSON object containing the username.}
    
    \begin{verbatim}
                        
        {
            "username": String
        }

    \end{verbatim}

                \item \textit{response parameters: there is no response body.}

            \end{itemize}

        \clearpage
        \subsection[User unregister]{User unregister}

            \vspace{10pt}

            This request, called \textbf{user\_unregister}, can be used to perform the unregistration of the user by passing the JSON encoded credentials in the body of the request.

            \begin{itemize}

                \item \textit{url endpoint: \ldots/userservices/logout.php}
                \item \textit{method: DELETE}
                \item \textit{possible HTTP status codes:}

                    \begin{itemize}

                        \item \textit{200 ``OK'': request was successful.}
                        \item \textit{400 ``Bad Request'': request had missing or malformed parameters.}
                        \item \textit{401 ``Unauthorized'': user doesn't exist.}
                        \item \textit{403 ``Forbidden'': wrong password for the specified username.}
                        \item \textit{404 ``Not found'': could not find the specified service.}
                        \item \textit{500 ``internal server error'': error in negotiation with the database.}

                    \end{itemize}

                \item \textit{request parameters: JSON object containing the credentials.}
    
    \begin{verbatim}
                        
        {
            "username": String,
            "password": String
        }

    \end{verbatim}

                \item \textit{response parameters: there is no response body.}

            \end{itemize}

\end{document}